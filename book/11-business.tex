\chapter{Bussiness}


% ---------- Bussiness Sample Metric ----------
\clearpage
\thispagestyle{businessstyle}
\section{Bussiness Sample Metric}
\subsection{Bussiness Sample Metric}

% ---------- CR ----------
\clearpage
\thispagestyle{businessstyle}

\section{CR}

\subsection{Conversion Rate}
In machine learning, Conversion Rate is the percentage of users who complete a desired action after being targeted by a model's recommendations or predictions.
For instance, in e-commerce conversion rate measures the percentage of users who purchase after receiving product recommendations from a machine learning model.

% equation
\begin{center}
    \tikz{
        \node[inner sep=2pt, font=\large] (a) {
            {
                $\displaystyle
                 Conversion \, Rate =\frac{Number \, of \,Conversions}{Number \, of \, Visits}\times100
                $
            }
        };
    }
\end{center}

\vspace{-10pt}

A high conversion rate indicates that the model successfully identifies and engages relevant users, leading to effective recommendations and increased sales.

\textbf{When to use Conversion Rate?}

Use conversion rate to evaluate marketing effectiveness, website performance, and user experience optimization, especially during campaigns, A/B testing, and product launches.

\coloredboxes{  % Ensure that this command is defined in your preamble
\item Provides a direct, measurable way to assess the effectiveness of machine learning models, marketing campaigns, or recommendation systems.
\item Versatile across various domains, including e-commerce, online advertising, and product recommendations.
}
{
\item Ignores positive user engagement that doesn’t lead to immediate conversions.
\item Optimizing for conversion rate can sometimes favor short-term wins over long-term user experience.
}

\textbf{Other related metrics}

Other related metrics to Conversion Rate include Bounce Rate and Return on Investment (ROI).

